\chapter{国内外研究现状与研究趋势}

区块链与智能合约技术概述,引出隐私保护问题和协同计算技术研究现状

\section{面向交易数据的区块链隐私保护方案概述}

从区块链的角度介绍隐私保护方面的技术分类

\subsection{区块链隐私问题}

\begin{enumerate}
    \item 身份隐私
    \item 交易隐私
\end{enumerate}

\subsection{基于数据失真的隐私保护}

\subsection{基于加密机制的隐私保护}

\subsection{基于限制发布的隐私保护}

\subsection{技术对比与评估}

加入表格,从目标场景、安全性、效率几个角度进行对比分析

\section{链上链下协同技术概述}

根据状态转变检查方式区分

\subsection{状态检查在链上完成}

\subsection{状态检查在链下完成}
链下存储,链上引用

\subsection{状态检查基于链上验证}

zokrates
CoCo
Ekiden

\subsection{状态检查基于验证与链上计算}

Scalable and Privacy-preserving Design of
On/Off-chain Smart Contracts
PrivacyGuard
Zether
Hawk

\subsection{技术对比与评估}

加入表格,从对链的影响、效率、隐私性、目标场景几个角度进行对比分析

\section{链下计算技术}
主要从以下四个角度介绍,其中重点分析可验证计算的优缺点,强调约束构建流程耗时带来的问题

\subsection{安全多方计算}

\subsection{激励驱动设计}

\subsection{可信执行环境}

\subsection{可验证计算}

1. 可验证计算概念</br>
2. 可验证计算分类</br>
zk-STARKs</br>
BulletProofs</br>
zk-SNARKs</br>
a. 基于zk-SNARKs的智能合约框架</br>
zkay Hawk ZEXE </br>
b. 基于zk-SNARKs的语法规范</br>

\subsection{技术对比与评估}

加入表格,从安全性、隐私性、问题描述方式、计算与验证原理几个角度对比分析

