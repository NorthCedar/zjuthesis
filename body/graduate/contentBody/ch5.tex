\chapter{基于约束分析的链下计算优化策略}
在第三章中对链下计算方案架构进行整体介绍,包括证明模块和预处理模块。本章主要讨论预处理模块的优化策略

\section{设计动机与目标}
针对circom方案进行详细介绍,电路的约束生成与通过重编译构造用于计算的可执行文件,指出该方案的低效

\subsection{语法一致性}
与circom电路语法保持一致,避免密码学工具库的再开发

\subsection{证明有效性}
输出结果为r1cs结构,完整保留电路中的约束内容,适用于常见证明算法

\subsection{高效性}
加快约束的构建和见证生成执行速度

\section{语义分析}

\section{电路约束构造}
在解析生成电路约束时,保留未知变量及其上下文

\section{电路约束约减}
在构建时提前约减,减少后续统一约减过程的工作量

\section{见证计算分析}
1. 约束系统转为最终计算公式
2. 使用分组并行加快计算速度
