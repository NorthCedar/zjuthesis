\chapter{绪论}

\section{研究背景}
区块链技术允许无需可信第三方参与的情况下在互不信任的实体之间达成一致。在区块链上提交的交易将通过共识被保存在分布式的节点上并保证其历史无法被更改,具备了不可伪造、公开透明、可追溯等特性。目前,在计算机科学领域和经济学领域都针对这一技术进行了广泛的研究和应用,而许多其他行业和领域也认识到了其特性带来的潜在价值并积极开展与区块链技术相结合的研究工作。

2008年,中本聪首先将区块链技术应用于构建比特币\cite{bitcoin}。该工作将分布式系统设计与经济激励相结合从而构建了一个不依赖于可信第三方的转账交易系统。其将区块链作为一个可追加的交易记录模块,以不可变的方式记录通过对等网络共识排序后的交易信息。这一设计将每个参与的节点都变为了系统内所有交易的验证者,账本被重复地保存在了分布式的节点中。

然而受限于其初始设计,比特币的应用场景较为狭窄。一些工作\cite{Namecoin, coloredcoins}尝试在比特币上实现与转账交易无关的应用需求,但往往需要付出极大的代价或将这类逻辑放在外部网络中实现。对此为将比特币在转账和金融领域的应用扩展到更广泛的领域,Vitalik提出了通用化的区块链平台:以太坊\cite{Ethereum}。其不再只是固定地执行转账交易逻辑,而是允许执行用户自定义的程序代码,我们称之为智能合约,从而使得整个系统变为了一个分布式的计算平台。

基于新的平台和智能合约的概念,区块链中也不再只是存储转账和数字资产信息,而可以存放任意应用数据。实际上应用使用者并不会在意应用是否被部署在分布式系统上,这一属性并不能直接为他们带来经济利益。然而,区块链带来的安全性、可靠性和去中心化等优势使得其成为许多应用的理想底层平台。研究人员提出了许多可以使用区块链特性的新应用场景:基于区块链的企业供应链溯源\cite{zhang2020blockchain},无需第三方参与的版权保护和认证\cite{savelyev2018copyright},以及自动化执行的防篡改电子签约\cite{9380170}等。

遗憾的是,虽然基于区块链的应用场景丰富,但是目前并没有太多的应用得以大规模部署实现。在技术实现上,隐私保护成为摆在区块链主流应用前的一个亟待解决的难题。数据隐私是每个应用平台都需要考虑的重要属性,使用者总是希望服务提供者能够保护自己的私人数据。对于个人用户来说或许是健康状况、财务信息等。而对于商业主体来说数据隐私也同样重要,例如供应链信息或内部技术等的泄露可能造成不可预计的经济损失。在《通用数据保护条例》\cite{GDPR}中已将隐私保护作为数字基础设施必须实现的强制性要求。

在中心化的应用程序中,用户信息都被合理存放在集中管理的数据库中。信息由可信任的应用提供商处理,并受到数据保护法规的约束。隐私保护的主要实现方式是通过访问控制进行隔离。然而这一方案与现在的区块链常规执行逻辑是相违背的:在当前的区块链网络中,所有节点都必须重复地执行每一笔交易,从而确保区块链中执行的正确性。而执行交易意味着交易的内部信息对于节点必须是公开可见的且内容被完整存储在了每个节点上。因此为避免数据在网络中泄露,隐私保护的内容无法在区块链上作存储和计算。对此,满足区块链应用所需的隐私保护要求成为区块链技术研究的一个关键问题。

\section{存在的问题与研究现状}
\paragraph{存在的问题} 首先为了进一步深入了解和论证区块链应用在隐私保护上存在的需求,下面我们将介绍当前存在的具体使用场景,并将其归约为一个通用的使用模式。

\begin{enumerate}
\item 金融KYC监管

基于《金融机构反洗钱规定》等法规要求,客户在进行大额交易时应持有金融机构出具的客户身份认证信息(KYC)。这一流程可以被描述为:当用户想要办理业务时将尝试向指定的金融机构申请其KYC证明,当机构提交KYC并被业务方验证通过后,后续的业务流程将继续执行。

区块链作为分布式账本具备公开透明和不可伪造等特性,将这一流程部署在区块链平台上有助于优化业务的自动化处理,推动线上金融体系的安全发展。当金融机构和监管平台作为节点加入区块链平台时,机构出具的KYC证明可以被实时监管,同时一旦KYC被上传,参与区块链的多方机构也将同步收到更新,有助于提高客户业务的办理效率。文献\cite{fi12020041}提出了使用IPFS的KYC区块链应用设计方案。虽然在链上存储时使用的是IPFS提供的文件标识符,但是在实际使用时其他机构仍将从IPFS中获取到客户的具体信息。然而KYC中往往包含了客户的敏感信息,因此如何在隐私保护的要求下在区块链中创建KYC证明并实现这一流程成为问题。

\item 医疗数据分析

医疗原始数据一般由单个医疗机构组织独立保存,在医疗数据分析中往往面临数据样本不足、数据维度缺失等问题。然而,直接将医疗数据提供给数据分析方用于计算将导致隐私信息的泄露和数据所有权的流失,这是患者与医疗机构都不愿看到的结果。

因此对医疗数据的使用权和所有权进行分离,多方的医疗数据在本地完成计算后输出脱敏的增值数据,有助于在医疗分析取得更大价值的同时降低隐私泄露的风险。由于计算过程涉及多方,基于区块链的分布式医疗分析应用有助于消除存在的信任问题,并通过部署的智能合约统一组织各计算参与方完成计算协议。交易在链上的留存也可作为数据使用追溯的凭证。然而基于目前的智能合约隐私保护方案,无法在协调各方计算与隐私数据链下计算之间达成一致。
    
\item 物流数据采集追踪

数据所有权与使用权分离的情况同样发生于区块链冷链物流管理中。相对于一般产品,冷链产品在流转的各个环节必须处于特定环境中以避免受到污染或变质。区块链应用有助于冷链物流合作的透明化,降低物流成本。

在一些跨度大,涉及物流主体较多的业务场景中,企业之间存在信息交互与共享需求,然而出于自身利益的考虑,物流企业与物流发起方并不愿意对外公开财务或订单具体内容。显然需要通过“链上业务发起-链下物流数据处理-链上验证与业务流转”完成对物流数据的隐私保护与物流业务的有效追踪。

\end{enumerate}

可以看到上述的具体问题都存在共同的特点,即整个业务流程中包含多方的数据隐私保护需求,需要一套完善的机制补足节点间的交互同时提供对于业务参与方的数据隐私保护能力。目前大量研究也针对区块链应用的隐私问题提出了相关的解决思路,接下来我们将对相关工作进行初步的介绍并检查他们是否具备解决前述问题的能力。
\paragraph{研究现状}
比特币\cite{bitcoin}作为第一个基于区块链的应用系统创建了一个去中心化的分布式电子货币系统。协议基于同态加密和决定交易顺序的共识机制共同实现交易安全。然而这一应用并未提供用户身份的匿名和交易内容的隐藏。Dash\cite{Dash}使用混币技术在主节点上将多笔交易合成一笔,从而使攻击者无法获取单笔交易的关联信息。随后为解决Dash主节点易受攻击和恶意混币参与者的问题,Monero\cite{monero}提出了一种依赖于主节点的混合加密方案。其使用隐蔽地址来解决交易间的关联性,并采用环签名机制解决交易双方的匿名问题。然而环签名机制依然需要与其他参与者的公钥共同工作,仍存在潜在的恶意参与者攻击。Zcash\cite{Zcash}则基于零知识证明技术提出了隐私性更好的加密货币工作方式,用户通过mint和pour操作向链上提供证明实现匿名的转账。可以看到这些加密货币基于不同技术实现了对交易金额,转账账户等信息的隐藏。但是这些工作都只提出了针对货币交易的细化研究,通用化的区块链应用隐私保护仍需要进一步探讨。

Hyperledger Fabric\cite{androulaki2018hyperledger}、Quorum\cite{Quorum}等联盟链似乎在区块链智能合约的隐私保护方面有天然的优势。其从架构上阻止了非授权用户对数据的访问,从而保证隐私数据不会被泄露到外部。然而,在内部网络中,不同的参与方依然共享所有数据信息,交易依然需要在所有节点上重复执行。这使得当某一方想要对其他参与者隐藏自己的隐私数据时,联盟链机制面临了同样的困局。

同样的问题也发生在链下签名机制中,交易参与方共同对链上状态的更新作签名,当且仅当签名被验证通过时该状态被接受。而交易的参与者通过链下网络进行合作:一方对状态完成计算并签名后将信息发送给其他参与方,其他参与方如果同意这一变更则也提供自己的签名并返回相应信息。文献\cite{poon2016bitcoin}在比特币生态中实现了这一方案,而Raiden\cite{}则给出了基于以太坊的实现。对于不涉及交易的节点来说,交易的内容将无法获取,但是参与者必须互相信任对方不会泄露交易的内容。

IPFS\cite{benet2014ipfs},Swarm\cite{swarm}和Filecoin\cite{Filecoin}等技术提供了基于内容寻址和区块链的分布式存储平台和内容分发服务,这有助于提供区块链应用的隐私保护。可以将智能合约中的真实数据转换为存储系统中对应的唯一标识符,并对其设定访问权限来提供隐私保护能力。然而这一方案存在的主要问题是:使用了外部的存储系统意味着存在数据可用性问题,一旦数据在外部丢失或存储系统不可用将导致相应的区块链应用也无法正常工作。同时受限于链上的重复执行,这一方案将不允许外部数据对链上状态造成影响。

另一种方案是基于链下可验证计算来提供隐私保护的智能合约。基于可验证计算可以在链上提供关于链下计算的正确性证明的同时而不泄露相关的隐私数据。虽然这一技术在隐私保护上的潜力较大,但是目前的基于这一方案的实现仍然较少且局限于特定的使用场景,例如前述的Zcash\cite{Zcash}。此外的一些研究工作都存在各自的局限性:Hwak\cite{7546538}及基于其他链下计算技术的工作如Arbitrum\cite{217511}和Ekiden\cite{2018arXiv180405141C}等,则是基于可信赖的管理者角色或是可信硬件等强大的信任模块提供了解决方案。ZoKrates\cite{8726497}从隐私保护合约开发者的角度出发提供了一个基于可验证计算的链下计算工具库,遗憾的是并没有为其设计一个通用的使用框架。Zkay\cite{2020arXiv200901020B}和ZeeStar\cite{9833732}基于这一工作实现了将合约中的隐私变量和计算自动转译为其提供的可验证计算原语的形式,然而这一思路暴露了内存地址存在隐私泄露的隐患\cite{10.1145/3548606.3560622}。此外,这些工作提供的框架都默认了合约调用者即为链下计算的创建者,而没有过多考虑多方参与下的隐私合约交互问题。

综上,对于区块链应用面临的隐私问题仍需要新的解决方案。
\section{研究目的与意义}

为了明确定义本文的研究范围和方向,根据前述对于去中心化的区块链应用面临的隐私保护问题的讨论,我们对当前研究存在的问题进行归纳:

背景一:隐私保护是区块链应用得以实现的关键条件,链下计算是一种在技术上解决智能合约隐私问题的有效机制。但是目前的智能合约隐私保护解决方案并不能较好适应多方参与的场景。

问题一:能否给出多方参与业务下的智能合约隐私保护模型?

背景二:多方参与的业务执行一般被分割为链上合约的相关功能接口并提供调用。然而在加入链下计算后,其链上合约将不再只是业务功能的实现还应包含对链下计算的验证和申请,这需要重新定义业务步骤间的协作关系并分析对链下计算的影响。

问题二:基于智能合约隐私保护模型是否存在一种安全且可行的协同工作机制从而支撑状态的更新?

背景三:构建的模型和机制只是从理论角度阐述了解决思路,现有技术手段是否满足这一设计尚不可知。

问题三:如何工程化实现上述的隐私保护解决方案并提高系统的运行效率?

\section{主要工作与创新点}

在这篇论文中,我们解决了上述提到的研究问题,从而显著提升了基于智能合约的区块链应用在隐私保护方面的扩展能力与应用价值。

1. 提出一个智能合约隐私保护模型,从而满足多方参与的自动化执行业务场景下的隐私需求,明确隐私合约执行过程中的各方角色,为协同计算工作机制的设计和系统实现打下基础。

2. 基于上述模型,我们详细设计了一种链上链下协同工作机制,使之适用于多方参与的自动化执行业务场景下。

3. 针对上述解决方案,我们介绍了其系统实现并从链下计算和协同交互的角度介绍系统实现中的优化策略,该策略有助于提高计算效率优化业务执行过程。

\section{论文组织结构}

