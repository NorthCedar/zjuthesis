\chapter{绪论}

\section{研究背景}

区块链技术允许互不信任的实体之间达成一致而无需额外的可信第三方参与。区块链的特点:公开、透明。(区块链技术介绍)
区块链最开始起源于比特币,后续也被广泛应用于构建点对点的去中心化支付系统。但这一应用仅仅局限于支付交易,无法支持一些可自定义的其他用例。(区块链发展背景,在金融转账方面)
对此涌现出一批通用区块链平台的发展:以太坊。从而提出了智能合约的概念,将区块链平台作为分布式的计算平台来执行用户自定义程序。例如:存证、摇号系统。(智能合约的目的、概念、工作方式)
然而数据隐私成为大量应用上链的主要问题。区块链交易的处理依赖于区块链节点,这要求交易中的所有数据和操作都对所有节点公开。这对于存在敏感数据的应用程序来说是不可接受的。(区块链中的隐私保护问题//本小结重点)
\cite{bell2018applications}
\section{研究现状}

对于智能合约的使用者并不会过多关心去中心化。可靠性、最低的信任要求和低成本成为选用区块链作为应用底层平台的直接优势,但是随之带来的智能合约隐私泄露问题成为了亟待解决的一大问题。例如,用户希望在使用应用时自己的健康状况、财务信息等得到保护。同样,在供应链中企业可能需要在数据共享的同时保持竞争优势。(问题复述)
在中心化的应用程序中,用户信息都被合理存放在集中管理的数据库中。信息由可信任的应用提供商处理,并受到数据保护法规的约束。隐私保护的主要实现方式是通过访问控制进行隔离。(中心化隐私保护解决方案)
但是在去中心化的区块链应用中,链上处理的数据必须是所有节点可用的。因此隐私保护不再能依靠数据隐藏实现,人们开始寻求技术层面的解决思路。同时链下的计算也有助于解决区块链应用的可扩展性问题。(链上链下协同的思想)
在转账方面有许多加密货币项目 ( zcash \ Monero),实现了对金额,转账账户等信息的隐藏。但是这些工作都只对货币交易进行不可复用的细化研究。(简单货币场景)
而在智能合约的隐私保护上,不同于加密货币,需要提供一种通用隐私保护策略。hawk引入代理人角色执行职能合约、Ekiden借助可信执行环境实现计算和共识的分离,PrivacyGuard提供了对远程隐私数据的访问;zokrates定义了扩展区块链应用的链下隐私计算。(引出目前的隐私保护协同计算方案与解决数据孤岛之间存在的问题,本小结重点)
综上,我们已经认识到对于区块链应用面临的隐私问题仍需要新的解决方案。为了进一步深入了解和论证区块链应用在多方面隐私保护上存在的需求,我们更详细地描述了两个去中心化区块链应用的具体背景。

\begin{enumerate}

\item 医疗数据分析

医疗原始数据一般由单个医疗机构组织独立保存,在医疗数据分析中往往面临数据样本不足、数据维度缺失等问题。然而,直接将医疗数据提供给数据分析方用于计算将导致隐私信息的泄露和数据所有权的流失,这是患者与医疗机构都不愿看到的结果。
因此对医疗数据的使用权和所有权进行分离,多方的医疗数据在本地完成计算后输出脱敏的增值数据,有助于在医疗分析取得更大价值的同时降低隐私泄露的风险。由于计算过程涉及多方,基于区块链的分布式医疗分析应用有助于消除存在的信任问题,并通过部署的智能合约统一组织各计算参与方完成计算协议。交易在链上的留存也可作为数据使用追溯的凭证。然而基于目前的智能合约隐私保护方案,无法在协调各方计算与隐私数据链下计算之间达成一致。
    
\item 物流数据采集追踪

数据所有权与使用权分离的情况同样发生于区块链冷链物流管理中。相对于一般产品,冷链产品在流转的各个环节必须处于特定环境中以避免受到污染或变质。区块链应用有助于冷链物流合作的透明化,降低物流成本。
在一些跨度大,涉及物流主体较多的业务场景中,企业之间存在信息交互与共享需求,然而出于自身利益的考虑,物流企业与物流发起方并不愿意对外公开财务或订单具体内容。显然需要通过 链上业务发起-链下物流数据处理-链上验证与业务流转 完成对物流数据的隐私保护与物流业务的有效追踪。

\end{enumerate}

\section{研究目的与意义}

为了明确定义本文的研究范围和方向,根据前述对于去中心化的区块链应用面临的隐私保护问题的讨论,我
们对当前研究存在的问题进行归纳:

背景一:隐私保护是区块链应用得以实现的关键条件,链下计算是一种在技术上解决智能合约隐私问题的有
效机制。但是目前对于链上链下协同计算的框架并不能较好适应隐私数据使用权与所有权分离的场景。

问题一:能否给出基于隐私数据使用权与所有权分离背景下的协同计算框架?

背景二:在基于智能合约的区块链应用中需要实现链上状态的更新。那么对应于问题一解决的协同计算框架
中,链上合约与链下计算的交互逻辑将会成为状态更新的重要条件。

问题二:在这一框架下是否存在一种可行且高效的协同工作机制从而支撑链上状态的更新?

背景三:在协同计算框架中另一不可或缺的部分即为链下计算。可验证计算不需要过多的硬件支持即可以实
现隐私数据的隐藏和计算验证,然而计算的效率问题是可验证计算应用受限的一大原因。

问题三:如何提高链下计算的运行效率?

\section{主要工作与创新点}

在这篇论文中,我们解决了上述提到的研究问题,从而显著提升了基于智能合约的区块链应用在隐私保护方面的扩展能力与应用价值。

1. 提出一个用于智能合约隐私保护的链上链下协同计算框架,从而满足隐私数据使用权与所有权分离场景下的隐私需求,明确隐私合约执行过程中的各方角色,为链上状态更新协议与链下计算模式的设计打下基础。

2. 基于上述框架,我们详细设计了一种链上状态更新协议,对链上链下协同工作机制进行丰富和完善。

3. 针对框架中的链下计算模块,我们给出了一种基于zk-SNARKs链下计算的优化策略,该策略有助于提高链下问题约束构建与计算过程的执行性能。

\section{论文组织结构}

