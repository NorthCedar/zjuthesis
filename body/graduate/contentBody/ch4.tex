\chapter{基于协同计算的链上状态变更协议}
重点对协议的每次交互内容和完整性角度作阐述,分析每步可能出现的异常及对协议的影响。

\section{设计动机与目标}
介绍现有方案的详细实现细节,在通用性(privacyGuard)和隐私保护的可用性(hybrid On/Off-chain)上存在问题
1. 链下计算请求未上链,无法留证,导致数据所有方无法信任请求方
2. 依赖于状态变更后的链下计算无法自动完成
\subsection{可用性}
对隐私保护问题的解决

\subsection{通用性}
对使用方与所有方分离场景的适用性

\subsection{高效性}
交互次数的降低

\section{链上状态变更协议架构}
由链上构建链下计算任务,通过交易方式完整记录所有链上链下交互内容,链下计算节点的回调完成合约的后续状态变更

\section{针对迭代计算的优化策略}
在完成链上验证时附带下一次计算任务信息,减少链上链下交互次数

\section{协议安全性分析}
在完成链上验证时附带下一次计算任务信息,减少链上链下交互次数

\subsection{隐私计算验证}

\subsection{交易信息验证}