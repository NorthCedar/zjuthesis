\chapter{隐私保护的链上链下协同计算架构与协议优化}
重点对协议的每次交互内容和完整性角度作阐述,分析每步可能出现的异常及对协议的影响。
\section{设计动机与目标}
介绍现有方案的详细实现细节,在通用性(privacyGuard)和隐私保护的可用性(hybrid On/Off-chain)上存在问题
1. 链下计算请求未上链,无法留证,导致数据所有方无法信任请求方
2. 依赖于状态变更后的链下计算无法自动完成
\subsection{可用性}
对隐私保护问题的解决

\subsection{通用性}
对使用方与所有方分离场景的适用性

\subsection{高效性}
交互次数的降低

\section{协同计算架构}
框架主要分为三层,在链上合约和链下计算之外引入交互验证作为缓冲层。

\section{业务逻辑层}
主要包括链上合约的执行逻辑、链上验证的调用、对链上状态的更新

\section{协同计算层}
主要包括验证密钥的保存、链上验证、链下计算任务的构建

\section{链下计算层}
主要包括对链上任务的响应与回调、链下计算、隐私数据的自动获取、计算节点的可用性保证</br>
链下计算:为实现zk-SNARKs计算方案,需要证明模块和预处理模块</br>
//局部架构图主要内容:
1. 程序编译
2. 见证计算
3. 可信设置
4. 证明生成
5. 证明验证

\section{链上状态变更协议架构}
由链上构建链下计算任务,通过交易方式完整记录所有链上链下交互内容,链下计算节点的回调完成合约的后续状态变更

\section{针对迭代计算的优化策略}
在完成链上验证时附带下一次计算任务信息,减少链上链下交互次数
