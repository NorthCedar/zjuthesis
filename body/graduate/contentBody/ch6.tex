\chapter{实验与分析}

\section{实验背景}
介绍实验机器、测试策略、每个实验的作用、用到的测试工具

\section{链上状态变更协议的实验设计与结果分析}

\subsection{链上创建计算}
1. 性能:交易payload大小、单次耗时
2. 压测:tps
链下计算的发起是一个常数复杂度的调用,给出平均值即可

\subsection{链上验证场景}
1. 性能:交易payload大小、单次耗时
2. 压测:tps

链上验证与链上计算两种方案的对比:选取通用的计算问题提供电路版本和合约实现版本,在4节点集群上分别测试

公共输入规模对比:公共输入在10,100,1000 数量级上

\subsection{多轮状态变更}
1. 性能:一次合约逻辑生成的交易数、发起的事件数、总流程耗时
2. 压测:tps

本协议、不提供验证、争论解决方案(不发生争论、发生争论)对比:选取通用的计算问题提供电路版本和合约实现版本,在4节点集群上分别测试

非递归与优化后方案对比:
//流程处理图</br>

```
n = 1, n = 20, n = 100, n = 500, n = 1000
for i < n {
   a+b=c
   a = c
   b = 1
}
set value to c
```


\section{链下计算的实验设计与结果分析}
1. 内存开销
2. 时间开销
3. cpu开销

分别使用本方案、bellman方案、circom方案:取大中小三类门数递减的不同规模电路运行计算命令

\section{协同计算框架的实验设计与结果分析}
对第一章的医疗问题进行详细描述,指出解决思路

\subsection{应用设计}
合约设计、隐私数据结构与存取方式

流程图主要设计:
1. 链下计算节点启动
2. 链下计算程序部署
3. 链上验证合约生成
4. 链上业务合约与验证合约部署
5. 链下计算节点事件监听
6. 发起业务合约交易
7. 各模块执行结果与链上状态变更查询

\subsection{结果分析}
1. 性能:总流程耗时
2. 压测:tps
3. 可扩展性:接入节点数量对总流程耗时的影响

\section{实验总结}
1. 应用范围与功能实现:医疗问题

2. 证明带来的影响:无证明合约(吞吐量,延迟)
3. 同类算法对比:吞吐量
4. 扩展性:多节点4-10(吞吐量,延迟)
